\section{Задача 7}
% \onlygroup{Мишунина}
Найти подстроку в тексте. При сравнении строк, если несовпадений символов было не более~$k$, строки считаются равными. $\O(nk \log n)$.

\subsection{Решение}
Как проверить, что строка подходит под образец:

\noindent
\begin{minipage}{\hsize}
\begin{algorithmic}
    \Function{Compare}{$s, r, k$}
        \Comment{$|s| = |r|$}
        \If{$k = 0$}
            \State \Return $s = r$
            \Comment{$\O(1)$ по хэшированию}
        \Else
            \State $p \gets \Call{LongestPrefix}{s, r}$
            \Comment{$\O(\log n)$}
            \If{$|p| = |s|$}
                \State \Return $1$
            \Else
                \State \Return $\Call{Compare}{s[|p| + 1; |s|], r[|p| + 1; |r|], k - 1}$
            \EndIf
        \EndIf
    \EndFunction
\end{algorithmic}
\end{minipage}

Можно заметить, что после предподсчёта для хэширования подстрок
функция \textsc{Compare} будет работать за $\O(k \log n)$.
Всего существует $\O(n)$ подстрок, которые могут совпасть с искомой по длине,
следовательно, сравнив их все, получаем время работы $\O(nk \log n)$.

Искомый алгоритм найден.
