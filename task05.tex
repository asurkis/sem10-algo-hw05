\section{Задача 5}
\begin{enumerate}
    \item Найти количество подпалиндромов строки. $\O(n \log n)$.
    \item Найти максимальный подпалиндром строки. $\O(n)$.
\end{enumerate}

\subsection{Решение}
\subsubsection{Пункт (a)}
Построим в дополнение к строке $s$ развёрнутую строку $r$.
Очевидно, $r[n + 1 - i] = s[i]$.
Тогда $r[n + 1 - i; n + 1 - j] = s[i; j]$.
Совершим предподсчёт для вычисления хэшей подстрок $s$ и $r$ за $\O(1)$,
тут критично, чтобы это были именно одинаковые хэш-функции.

Пусть в $s$ есть какой-то палиндром с центром в $i$ и длины $2k+1$.
Тогда он будет повторён в $r$ с центром в $n + 1 - i$.
Очевидно, он будет также содержать $k$ меньших палиндромов с центром в $i$.

Если же есть какой-то палиндром длины $2k + 2$,
расположенный в $s[i - k - 1; i + k]$,
то он повторён в $r[n + 1 - i - k; n + 2 - i + k]$.

Тогда пройдёмся по всем индексам $i=1,\ldots,n$
и найдём наибольшие общие префиксы (за $\O(\log n)$) у:
\begin{itemize}
    \item $s[i; n]$ и $r[n + 1 - i; n]$ --- обозначим за $p(i)$
    \item $s[i; n]$ и $r[n + 2 - i; n]$ --- обозначим за $q(i)$
\end{itemize}

$p(i)$ --- половина (включая середину) максимального палиндрома нечётной длины,
$q(i)$ --- половина максимального палиндрома чётной длины.
Тогда общее количество палиндромов:
\[ \sum_{i=1}^n \Bigl( |p(i)| + |q(i)| \Bigr) \]

Если в строке вообще нет палиндромов длины хотя бы 2,
то $|p(i)| = 1$ и $|q(i)| = 0$.

Очевидно, с хэшированием алгоритм работает за $\O(n \log n)$
и находит количество палиндромов.

Искомый алгоритм найден.
