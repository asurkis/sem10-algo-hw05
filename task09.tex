\section{Задача 9}
Пусть дана $w$-разрядная \texttt{RAM} машина. Для простоты можно считать
$w$ степенью двойки. Придумайте для такой машины детерминированный предподсчет за $\O(w)$ операций, который позволит по $w$-битному числу вида $2^k$ восстанавливать $k$ за $\O(1)$ операций.

\subsection{Решение}
Очевидно, что максимальная степень двойки,
которую можно записать в $w$ бит --- это $2^{w - 1}$.

Найдём последовательность де Брёйна для алфавита $\{0; 1\}$
и строк длины $n = \ceil{\log_2 w}$ (т.е. чтобы всего было $w \leq 2^n < 2w$ строк).
Построение последовательности де Брёйна займёт $\O(2^n) = \O(w)$ времени,
и она будет длины $(2^n + n - 1)$, т.к. она соответствует эйлерову
циклу в графе де Брёйна с $2^{n - 1}$ вершинами и $2^n$ рёбрами.
$2^n + n - 1 \in \O(2^n) = \O(w)$, а алфавит состоит из двух символов,
следовательно, последовательность де Брёйна можно упаковать в $\O(1)$ машинных слов.

Рассмотрим последовательность де Брёйна как длинное число.
Если мы умножим его на $2^k$, то, фактически, сместим последовательность на $k$ влево.
Поскольку размер последовательности --- $\O(1)$ машинных слов,
то мы можем за $\O(1)$ умножить её на число $2^k$ и получить уникальное по $k$
число в $n$ (а у нас $n < w$) разрядах, которые до умножения были старшими.

Тогда завершим предподсчёт такими умножениями на все возможные $2^k$,
и полученное число считаем индексом массива длины $2^n \in \O(w)$.
В соответствующую ячейку массива запишем соответствующее $k$.
Всего таких вычислений будет $w$, каждое из них --- за $\O(1)$,
тогда вместе с предыдущими шагами предподсчёт занимает $\O(w)$ времени.

Тогда ответ на запрос --- снова умножить число на последовательность де Брёйна
и произвести поиск по индексу в массиве, и достать оттуда $k$.
Очевидно, эти шаги выполняются за $\O(1)$, как и при построении.

Искомый алгоритм найден.
