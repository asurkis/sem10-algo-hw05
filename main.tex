\documentclass{article}

\usepackage[T2A]{fontenc}
\usepackage[utf8]{inputenc}
\usepackage[russian]{babel}
\usepackage{fullpage}
\usepackage{indentfirst}

\usepackage{amsmath}
\usepackage{amsfonts}
\usepackage{amsthm}
\usepackage{float}
\usepackage{tikz}

\usepackage[outputdir=out]{minted}
%\usepackage{algorithmicx}
\usepackage{algpseudocodex}
%\usepackage{algorithm2e}
\usepackage{hyperref}
\usepackage{enumitem}
\usepackage{array}

%\setlength{\parindent}{1.25cm}
%\renewcommand{\baselinestretch}{1.5}
\setlength{\parskip}{6pt}

\begin{document}
    \theoremstyle{definition}
    \newtheorem*{definition}{Определение}
    \newtheorem{theorem}{Теорема}
    \newtheorem{statement}{Утверждение}
    \newtheorem{lemma}{Лемма}

    \newcommand{\N}{\mathbb{N}}
    \newcommand{\R}{\mathbb{R}_{>0}}
    \renewcommand{\O}{\mathcal{O}}
    \renewcommand{\o}{o}
    \newcommand{\Const}{\mathit{Const}}
    \newcommand{\Mod}{~\text{mod}~}
    \newcommand{\thus}{\Rightarrow}
    \renewcommand{\H}{\mathcal{H}}
    \newcommand{\Prb}[1]{\underset{#1}{\textit{\textbf{Pr}}}}

    \newcommand{\paren}[1]{\left ( #1 \right )}
    \newcommand{\brackets}[1]{\left [ #1 \right ]}
    \newcommand{\braces}[1]{\left \{ #1 \right \}}
    \newcommand{\floor}[1]{\left \lfloor #1 \right \rfloor}
    \newcommand{\ceil}[1]{\left \lceil #1 \right \rceil}
    \newcommand{\abs}[1]{\left | #1 \right |}

    \hfill
    \begin{tabular}{ll}
        Студент: & Антон Суркис \\
        Группа:  & M4141        \\
        Дата:    & \today       \\
    \end{tabular}
    \hrule

    \begin{definition} Семейство хеш-функций $\H = \{ h : X \rightarrow Y \}$ называется
        \begin{itemize}
            \item универсальным, если:
            $$\forall x_1, x_2 \in X, x_1 \neq x_2: \Prb{h \in \H}\left[ \; h(x_1) = h(x_2) \; \right] \le \frac{1}{|Y|}$$

            \item $k$-независимым, если
            для любых различных $x_1, x_2, \cdots, x_k \in X$, для любых, возможно, совпадающих $y_1, y_2, \cdots, y_k \in Y$
            выполняется:
            $$\Prb{h \in \H}\left[ \; \bigwedge_{i=1}^k h(x_i) = y_i \; \right] = \frac{1}{|Y|^k}$$
        \end{itemize}
    \end{definition}

    \section{Задача 1}
\begin{enumerate}[label=(\alph*)]
\item Докажите, что любое 2-независимое семейство хэш-функций является универсальным.
\item Докажите, что любое $(k+1)$-независимое семейство хэш-функций является $k$-независимым.
\end{enumerate}

\subsection{Решение}
\subsubsection{Пункт (a)}
Пусть $\H$ --- 2-независимое семейство хэш-функций. Тогда
\begin{gather*}
    \forall x_1, x_2 \in X, x_1 \neq x_2 : \Prb{h \in \H} \left[ h(x_1) = h(x_2) \right] = \\
    = \Prb{h \in \H} \bigvee_{y \in Y} \left[ \Bigl( h(x_1) = y \Bigr) \land \Bigl( h(x_2) = y \Bigr) \right] = \\
    = \sum_{y \in Y} \frac{1}{|Y|^2} = \frac{1}{|Y|} \leq \frac{1}{|Y|}
\end{gather*}

То есть $\H$ --- универсальное семейство хэш-функций.
\qed

\subsubsection{Пункт (b)}
Пусть $\H$ --- $(k + 1)$-независимое семейство хэш-функций.
Возьмём некоторые $x_1,\ldots,x_k$ и произвольное $x_{k + 1}$.
При $y_i \neq y_j$ события $h(x_{k + 1}) = y_i$ и $h(x_{k + 1}) = y_j$ несовместны,
поэтому
\[ \Prb{h \in \H} [h(x_{k + 1}) = y_i \lor h(x_{k + 1}) = y_j] = \Prb{h \in \H} [h(x_{k + 1}) = y_i] + \Prb{h \in \H} [h(x_{k + 1}) = y_j] \]

Очевидно, что
\[ \Prb{h \in \H} \bigvee_{y \in Y} [h(x_{k + 1}) = y] = 1 \]

Тогда
\begin{gather*}
\Prb{h \in \H} \bigwedge_{i=1}^k [ h(x_i) = y_i ] = \\
\Prb{h \in \H} \bigvee_{y_{k + 1} \in Y} \bigwedge_{i=1}^{k+1} [ h(x_i) = y_i ] = \\
\sum_{y_{k + 1} \in Y} \frac{1}{|Y|^{k + 1}} = \frac{1}{|Y|^k} \\
\end{gather*}

То есть $\H$ --- $k$-независимое семейство хэш-функций.
\qed

    \section{Задача 2}
Придумайте строку длины $n$ над алфавитом $\{0, 1\}$, в которой $\Omega(n^2)$ различных подстрок.

    \section{Задача 3}
Даны две строки суммарной длины $n$. Найдите их самую длинную общую подстроку. $\O(n \log n)$.

\subsection{Решение}
Проведём предподсчёт для вычисления хэшей подстрок $s$ и $t$ за $\O(1)$.
Это займёт $\O(n)$ времени.
Далее двоичным поиском ищем максимальное $k$ такое, что у $s$ и $t$ существует общая подстрока длины $k$.
Проверка, что такая подстрока существует:
\begin{enumerate}
\item Вычисляем хэши всех подстрок $s$ длины $k$ и записываем в хэш-таблицу. $\O(n)$.
\item Вычисляем хэши всех подстрок $t$ длины $k$. Если соответствующих хэш найден в таблице,
то мы нашли подстроку длины $k$. Сохраним её на случай, если общих подстрок длиннее нет. $\O(n)$.
\end{enumerate}

Таким образом, мы ищем из $\O(n)$ возможных значений двоичным поиском,
одна проверка занимает $\O(n)$ времени, следовательно,
время работы всего алгоритма --- $\O(n \log n)$.

Искомый алгоритм найден.

    \section{Задача 4}
Даны строка $s$ длины $n$ и число $k$. Найдите в $s$ за время $\O(n)$ самую длинную подстроку,
представимую как степень  некоторой строки $x$, для которой $|x| = k$ (степенью строки
называют её конкатенацию с собой несколько раз).

\subsection{Решение}
За $\O(n)$ построим предподсчёт, позволяющий считать хэш строки за $\O(1)$.
Для каждого индекса $1,\ldots,n$ запишем, искали ли мы уже степень, начиная с этого индекса.
Очевидно, если мы уже нашли некоторую максимальную подстроку-степень,
то есть такую, которую нельзя расширить ни вправо, ни влево,
то смещение на $i \cdot k$ от её начала
даст нам более короткую степень.
Тогда алгоритм выглядит так:
\begin{algorithmic}
    \For{$i=1,\ldots,n$}
        \State $c_i \gets 1$
        \Comment{Раздаём ,,монетки`` подстрокам}
    \EndFor
    \State $i' \gets 0$
    \State $j' \gets 0$
    \For{$i=1,\ldots,n - k + 1$}
        \State $j \gets i$
        \While{$(c_j = 1) \land (j + k < n) \land (s[i ; i + k) = s[j ; j + k))$}
            \Comment{$\O(1)$ на итерацию}
            \State $c_j \gets 0$
            \Comment{Не более 1 итерации по каждому суффиксу, а всего их $n$}
            \State $j \gets j + k$
        \EndWhile
        \If{$j - i > j' - i'$}
            \State $i' \gets i$
            \State $j' \gets j$
        \EndIf
    \EndFor
    \State $s[i'; j' + k)$ --- ответ
\end{algorithmic}

Фактически, этот алгоритм проходит по всем
лево-максимальным подстрокам-степеням,
т.е. тем, которые нельзя расширить влево,
и по всем их расширениям вправо.
По каждой такой подстроке проход --- за $\O(1)$ благодаря хешированию,
и мы не рассматриваем повторно уже просмотренные суффиксы.
Тогда всего строк получается не более $n$,
и алгоритм работает за $\O(n)$.

Искомый алгоритм найден.

    \section{Задача 5}
\begin{enumerate}
    \item Найти количество подпалиндромов строки. $\O(n \log n)$.
    \item Найти максимальный подпалиндром строки. $\O(n)$.
\end{enumerate}

\subsection{Решение}
\subsubsection{Пункт (a)}
Построим в дополнение к строке $s$ развёрнутую строку $r$.
Очевидно, $r[n + 1 - i] = s[i]$.
Тогда $r[n + 1 - i; n + 1 - j] = s[i; j]$.
Совершим предподсчёт для вычисления хэшей подстрок $s$ и $r$ за $\O(1)$,
тут критично, чтобы это были именно одинаковые хэш-функции.

Пусть в $s$ есть какой-то палиндром с центром в $i$ и длины $2k+1$.
Тогда он будет повторён в $r$ с центром в $n + 1 - i$.
Очевидно, он будет также содержать $k$ меньших палиндромов с центром в $i$.

Если же есть какой-то палиндром длины $2k + 2$,
расположенный в $s[i - k - 1; i + k]$,
то он повторён в $r[n + 1 - i - k; n + 2 - i + k]$.

Тогда пройдёмся по всем индексам $i=1,\ldots,n$
и найдём наибольшие общие префиксы (за $\O(\log n)$) у:
\begin{itemize}
    \item $s[i; n]$ и $r[n + 1 - i; n]$ --- обозначим за $p(i)$
    \item $s[i; n]$ и $r[n + 2 - i; n]$ --- обозначим за $q(i)$
\end{itemize}

$p(i)$ --- половина (включая середину) максимального палиндрома нечётной длины,
$q(i)$ --- половина максимального палиндрома чётной длины.
Тогда общее количество палиндромов:
\[ \sum_{i=1}^n \Bigl( |p(i)| + |q(i)| \Bigr) \]

Если в строке вообще нет палиндромов длины хотя бы 2,
то $|p(i)| = 1$ и $|q(i)| = 0$.

Очевидно, с хэшированием алгоритм работает за $\O(n \log n)$
и находит количество палиндромов.

Искомый алгоритм найден.

\subsubsection{Пункт (b)}
Рассмотрим сначала нечётные палиндромы.
Заведём массив $d[i]$ --- количество палиндромов с центром в точке $i$.
Заведём также пару чисел $l$ и $r$ --- границы самого правого известного палиндрома.

Рассмотрим алгоритм:

\noindent
\begin{minipage}{\hsize}
\begin{algorithmic}
    \State $l \gets 0$
    \State $r \gets 0$
    \For{$i=1,\ldots,n$}
        \State $d[i] \gets 0$
    \EndFor
    \State $l \gets 0$
    \State $i' \gets 0$
    \For{$i=1,\ldots,n$}
        \If{$i \leq r$}
            \State $j \gets l + r - i$
            \Comment{Симметричная $i$ относительно центра палиндрома точка}
            \State $d[i] \gets \min(d[j]; j - l)$
            \LComment{
                Палиндромы с центром в $i$ как минимум не хуже,
                чем палиндромы с центром в $j$, не выходящие за пределы $[l; r]$}
        \EndIf
        \While{$d[i] < i \land i + d[i] \leq n \land s[i - d[i]] = s[i + d[i]]$}
            \State $d[i] \gets d[i] + 1$
        \EndWhile
        \If{$i + d[i] \geq r$}
            \State $l \gets i - d[i] + 1$
            \State $r \gets i + d[i] - 1$
        \EndIf
        \If{$d[i] \geq l$}
            \State $i' \gets i$
            \State $l \gets d[i] - 1$
        \EndIf
    \EndFor
    \State $[i' - l; i' + l]$ --- максимальный нечётный палиндром
\end{algorithmic}
\end{minipage}

Инициализация занимает $\O(n)$ времени.
Инициализация $d[i]$ в цикле --- $\O(1)$ на каждой итерации,
после этого либо палиндром с центром в $i$ вылезает за пределы $[l; r]$,
либо же $d[i]$ не увеличивается ни разу
(поскольку перевёрнутый палиндром с центром в $i$ уже был рассмотрен на позиции $j$),
что занимает $\O(1)$ времени.
$r$ может быть увеличен $\O(n)$ раз, поэтому весь проход занимает $\O(n)$ времени.

Аналогичным образом рассматриваются чётные палиндромы,
считаем, что их центры --- между символов и имеют индексы вида $k + \frac{1}{2}$,
тогда нужно делать соответствующую поправку $\pm \frac{1}{2}$.
Этот проход тоже будет занимать $\O(n)$ времени.

Искомый алгоритм найден.

%     % \section{Задача 6}
% % \onlygroup{Кравченко и Крыштаповича}
% Найдите $k$-й в лексикографическом порядке суффикс в строке.
% \begin{enumerate}
% \item $\O(n \log^2 n)$.
% \item $\O(n \log n)$.
% \end{enumerate}
 % Кравченко и Крыштапович
    \section{Задача 7}
% \onlygroup{Мишунина}
Найти подстроку в тексте. При сравнении строк, если несовпадений символов было не более~$k$, строки считаются равными. $\O(nk \log n)$.

\subsection{Решение}
Как проверить, что строка подходит под образец:
\begin{algorithmic}
    \Function{Compare}{$s, r, k$}
        \Comment{$|s| = |r|$}
        \If{$k = 0$}
            \State \Return $s = r$
            \Comment{$\O(1)$ по хэшированию}
        \Else
            \State $p \gets \Call{LongestPrefix}{s, r}$
            \Comment{$\O(\log n)$}
            \If{$|p| = |s|$}
                \State \Return $1$
            \Else
                \State \Return $\Call{Compare}{s[|p| + 1; |s|], r[|p| + 1; |r|], k - 1}$
            \EndIf
        \EndIf
    \EndFunction
\end{algorithmic}

Можно заметить, что после предподсчёта для хэширования подстрок
функция \textsc{Compare} будет работать за $\O(k \log n)$.
Всего существует $\O(n)$ подстрок, которые могут совпасть с искомой по длине,
следовательно, сравнив их все, получаем время работы $\O(nk \log n)$.

Искомый алгоритм найден.

%     \subsection*{Дополнительные задачи}
    \section{Задача 8}
Напишите на \texttt{Haskell} определение бесконечного списка
\texttt{Int}-ов (из $\{ 0, 1 \}$), представляющего собой последовательность
Туэ-Морса (последовательность из задачи 7 практики) $S_\infty$, такое, что:
\begin{itemize}
\item определение содержит не более 128 символов
\item можно использовать стандартные функции из модулей \texttt{Prelude} и \texttt{Data.List}
\item вычисление в нормальную форму \texttt{genericTake n} от списка занимает время $\O(n)$.
\end{itemize}

\subsection{Решение}
Код также доступен по ссылке:\\
\url{https://github.com/asurkis/sem10-algo-hw05/blob/main/hs/src/Lib.hs}

\inputminted{Haskell}{hs/src/Lib.hs}

\texttt{bin} --- двоичные длинные целые числа
\texttt{dec} --- десятичные длинные целые числа,
\texttt{str} --- строки.

Эти списки, определённые за 47, 54 и 83 символа соответственно
(65, 72 и 100, если считать объявления типов)
представляют из себя простые итераторы, с функцией за $\O(1)$ для \texttt{bin} и \texttt{dec},
поэтому \texttt{genericTake n} будет иметь сложность $\O(n)$.

Построение строки не будет иметь константную сложность из-за необходимости конкатенации.

Замена \texttt{Integer} на \texttt{Int} сильно ограничивает осмысленную длину списка.

Искомое решение найдено.

    \section{Задача 9}
Пусть дана $w$-разрядная \texttt{RAM} машина. Для простоты можно считать
$w$ степенью двойки. Придумайте для такой машины детерминированный предподсчет за $\O(w)$ операций, который позволит по $w$-битному числу вида $2^k$ восстанавливать $k$ за $\O(1)$ операций.

\end{document}
