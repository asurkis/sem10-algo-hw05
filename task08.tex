\section{Задача 8}
Напишите на \texttt{Haskell} определение бесконечного списка
\texttt{Int}-ов (из $\{ 0, 1 \}$), представляющего собой последовательность
Туэ-Морса (последовательность из задачи 7 практики) $S_\infty$, такое, что:
\begin{itemize}
\item определение содержит не более 128 символов
\item можно использовать стандартные функции из модулей \texttt{Prelude} и \texttt{Data.List}
\item вычисление в нормальную форму \texttt{genericTake n} от списка занимает время $\O(n)$.
\end{itemize}

\subsection{Решение}
Код также доступен по ссылке:\\
\url{https://github.com/asurkis/sem10-algo-hw05/blob/main/hs/src/Lib.hs}

\inputminted[firstline=3]{Haskell}{hs/src/Lib.hs}

Это выражение определяет искомую последовательность $S_\infty$ за 41 символ
(55, если считать объявление типа и переводы строк).

Из-за стратегии вычислений Haskell каждое значение списка, из которого строится ответ,
будет вычисляться не более 1 раза, при этом к моменту следующей итерации (iterate)
список-аккумулятор будет приведён в сильную нормальную форму,
следовательно, вычисление очередных $2^k$ элементов занимает $\O(2^k)$ времени,
и всего на первые $n$ элементов будет потрачено $\O(n)$ времени.

Искомое решение найдено.
