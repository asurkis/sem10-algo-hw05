\section{Задача 8}
Напишите на \texttt{Haskell} определение бесконечного списка
\texttt{Int}-ов (из $\{ 0, 1 \}$), представляющего собой последовательность
Туэ-Морса (последовательность из задачи 7 практики) $S_\infty$, такое, что:
\begin{itemize}
\item определение содержит не более 128 символов
\item можно использовать стандартные функции из модулей \texttt{Prelude} и \texttt{Data.List}
\item вычисление в нормальную форму \texttt{genericTake n} от списка занимает время $\O(n)$.
\end{itemize}

\subsection{Решение}
Код также доступен по ссылке:\\
\url{https://github.com/asurkis/sem10-algo-hw05/blob/main/hs/src/Lib.hs}

\inputminted{Haskell}{hs/src/Lib.hs}

\texttt{bin} --- двоичные длинные целые числа
\texttt{dec} --- десятичные длинные целые числа,
\texttt{str} --- строки.

Эти списки, определённые за 47, 54 и 83 символа соответственно
(65, 72 и 100, если считать объявления типов)
представляют из себя простые итераторы, с функцией за $\O(1)$ для \texttt{bin} и \texttt{dec},
поэтому \texttt{genericTake n} будет иметь сложность $\O(n)$.

Построение строки не будет иметь константную сложность из-за необходимости конкатенации.

Замена \texttt{Integer} на \texttt{Int} сильно ограничивает осмысленную длину списка.

Искомое решение найдено.
