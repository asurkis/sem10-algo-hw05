\section{Задача 3}
Даны две строки суммарной длины $n$. Найдите их самую длинную общую подстроку. $\O(n \log n)$.

\subsection{Решение}
Проведём предподсчёт для вычисления хэшей подстрок $s$ и $t$ за $\O(1)$.
Это займёт $\O(n)$ времени.
Далее двоичным поиском ищем максимальное $k$ такое, что у $s$ и $t$ существует общая подстрока длины $k$.
Проверка, что такая подстрока существует:
\begin{enumerate}
\item Вычисляем хэши всех подстрок $s$ длины $k$ и записываем в хэш-таблицу. $\O(n)$.
\item Вычисляем хэши всех подстрок $t$ длины $k$. Если соответствующих хэш найден в таблице,
то мы нашли подстроку длины $k$. Сохраним её на случай, если общих подстрок длиннее нет. $\O(n)$.
\end{enumerate}

Таким образом, мы ищем из $\O(n)$ возможных значений двоичным поиском,
одна проверка занимает $\O(n)$ времени, следовательно,
время работы всего алгоритма --- $\O(n \log n)$.

Искомый алгоритм найден.
